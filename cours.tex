% Copyright (C)  2016  Orestis Malaspinas.
%     Permission is granted to copy, distribute and/or modify this document
%     under the terms of the GNU Free Documentation License, Version 1.3
%     or any later version published by the Free Software Foundation;
%     with no Invariant Sections, no Front-Cover Texts, and no Back-Cover Texts.
%     A copy of the license can be downloaded from:
%     https://www.gnu.org/licenses/fdl.html.

\documentclass[a4paper,12pt]{book}
\usepackage[utf8]{inputenc}
\usepackage[french]{babel}
\usepackage{amsfonts,bm,amsmath,amssymb,graphicx,amsthm}
\usepackage{cancel}
\usepackage{mathtools}

\setlength{\parindent}{0pt}

\newcommand{\real}{\mathbb{R}}
\newcommand{\integer}{\mathbb{Z}}
\renewcommand{\natural}{\mathbb{N}}
\newcommand{\complex}{\mathbb{C}}
\newcommand{\zbar}{\bar{z}}
\newcommand{\dd}{\mathrm{d}}
\newcommand{\perm}{\mathrm{perm}}
\newcommand{\card}{\mathrm{card}}
\newcommand{\fh}{\hat{f}}
\newcommand{\gh}{\hat{g}}
\newcommand{\hh}{\hat{h}}
\renewcommand{\Re}{\mathrm{Re}}
\renewcommand{\Im}{\mathrm{Im}}
\newcommand{\pDeriv}[2]{\frac{\partial #1}{\partial #2}}
\newcommand{\pDerivTwo}[2]{\frac{\partial^2 #1}{\partial #2^2}}
\newcommand{\dDeriv}[2]{\frac{\dd #1}{\dd #2}}
\newcommand{\dDerivTwo}[2]{\frac{\dd^2 #1}{\dd #2^2}}
\newtheorem{definition}{Définition}
\newtheorem*{exemples}{Exemples}
\newtheorem*{exemple}{Exemple}
\newtheorem*{exercice}{Exercice}
\newtheorem*{exercices}{Exercices}
\newtheorem*{remarque}{Remarque}
\newtheorem{proprietes}{Propriétés}
\newtheorem{theoreme}{Théorème}

\title{Résumé du cours de Physique Générale}
\author{Orestis Malaspinas}

\begin{document}
\maketitle

\chapter{Mesures, incertitudes, et estimations}

Toues les sciences ``naturelles'' sont basées sur l'observation du monde qui nous entoure. 
Mais malgré le fait qu'on ait l'impression que le processus d'osbervation soit une suite simple:
observation, expérimentation, obtention de résultats, qu'on explique avec une théorie cela n'est pas vraiment le cas. En fait,
de façon proche à ce qui se passe dans les arts, les sciences sont un processus hautement créatif. 
En effet, lors d'une observation un scientifique ne décrit pas tout ce qu'il voit, 
mais sélectionne uniquement ce qu'il juge important
pour la compréhension et l'interprétation d'un phénomène. De plus, une fois sélectionné
le processus à observer, il convient de créer une expérience permettant de le mesurer de façon aussi précise que possible
pour pouvoir le décrire. La mesure tient donc une place centrale dans les sciences et se complète parfaitement avec la création de théories
qui permettent l'explication d'observations. Par ailleurs, toutes les théories ne sont pas le fruit
d'expériences (ou d'observations) mais ont souvent été le résultat de constructions de l'esprit. 
Dans ce cas les expériences, viennent confirmer (ou infirmer) les théories.

On voit donc que l'expérimentation est particulièrement importante pour tester des théories et s'assurer
que notre conception de la nature est aussi proche de la réalité que possible.


\end{document}
