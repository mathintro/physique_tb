% Copyright (C)  2016  Orestis Malaspinas.
%     Permission is granted to copy, distribute and/or modify this document
%     under the terms of the GNU Free Documentation License, Version 1.3
%     or any later version published by the Free Software Foundation;
%     with no Invariant Sections, no Front-Cover Texts, and no Back-Cover Texts.
%     A copy of the license can be downloaded from:
%     https://www.gnu.org/licenses/fdl.html.

\documentclass[a4paper,12pt]{article}
\usepackage[utf8]{inputenc}
\usepackage[french]{babel}
\usepackage{amsfonts,bm,amsmath,amssymb,graphicx,amsthm}
\usepackage{cancel}
\usepackage{mathtools}
\usepackage{caption}
\usepackage{hyperref}

\setlength{\parindent}{0pt}

\newcommand{\real}{\mathbb{R}}
\newcommand{\integer}{\mathbb{Z}}
\renewcommand{\natural}{\mathbb{N}}
\newcommand{\complex}{\mathbb{C}}
\newcommand{\zbar}{\bar{z}}
\newcommand{\dd}{\mathrm{d}}
\newcommand{\perm}{\mathrm{perm}}
\newcommand{\card}{\mathrm{card}}
\newcommand{\fh}{\hat{f}}
\newcommand{\gh}{\hat{g}}
\newcommand{\hh}{\hat{h}}
\renewcommand{\Re}{\mathrm{Re}}
\renewcommand{\Im}{\mathrm{Im}}
\newcommand{\pDeriv}[2]{\frac{\partial #1}{\partial #2}}
\newcommand{\pDerivTwo}[2]{\frac{\partial^2 #1}{\partial #2^2}}
\newcommand{\dDeriv}[2]{\frac{\dd #1}{\dd #2}}
\newcommand{\dDerivTwo}[2]{\frac{\dd^2 #1}{\dd #2^2}}
\newtheorem{definition}{Définition}
\newtheorem*{exemples}{Exemples}
\newtheorem*{exemple}{Exemple}
\newtheorem*{solution}{Solution}
\newtheorem{exercice}{Exercice}
\newtheorem{corrige}{Corrig\'e exercice}
\newtheorem*{remarque}{Remarque}
\newtheorem{proprietes}{Propriétés}
\newtheorem{theoreme}{Théorème}
\newcommand{\cm}{\mathrm{cm}}
\newcommand{\km}{\mathrm{km}}
\newcommand{\mm}{\mathrm{mm}}
\newcommand{\cd}{\mathrm{cd}}
\newcommand{\mol}{\mathrm{mol}}
\newcommand{\m}{\mathrm{m}}
\newcommand{\s}{\mathrm{s}}
\newcommand{\kg}{\mathrm{kg}}
\newcommand{\g}{\mathrm{g}}
\newcommand{\K}{\mathrm{K}}
\newcommand{\J}{\mathrm{J}}
\newcommand{\C}{\mathrm{C}}
\newcommand{\oC}{^\circ\C}
\newcommand{\A}{\mathrm{A}}
\newcommand{\N}{\mathrm{N}}
\newcommand{\atm}{\mathrm{atm}}
\renewcommand{\bar}{\mathrm{bar}}
\newcommand{\V}{\mathrm{V}}
\newcommand{\W}{\mathrm{W}}
\newcommand{\Pa}{\mathrm{Pa}}


\title{Exercices sur la température}
% \author{Orestis Malaspinas}
\date{}

\begin{document}
\maketitle

\begin{exercice}{Dilatation linéique/volumique}

Montrer qu’en première approximation, pour un solide sa dilatation volumique est telle que 
\begin{equation}
\beta=3\alpha.
\end{equation}
Indication: Écrire que l'allongement dans chaque direction peut s'écrire $L_1=L_0+\Delta L$, avec $\Delta L\ll 1$,
en déduire la variation
du volume, puis négliger les termes les plus petits.
\end{exercice}

\begin{corrige}{Dilatation linéique/volumique}
Nous pouvons exprimer un volume comme étant $V=L_1\cdot\L_2\cdot\L_3$ où les $L_i$ sont les longueur
dans chanque direction du solide. Lorsque nous changeons la température du solide nous avons 
\begin{equation}
L_i'=L_i+\Delta L_i.
\end{equation}
Et donc le volume devient 
\begin{align}
V'&=L_1'\cdot L_2'\cdot L_3'=(L_1+\Delta L_1)(L_2+\Delta L_2)(L_3+\Delta L_3),\nonumber\\
  &=L_1L_2L_3+\Delta L_1 L_2L_3+\Delta L_2 L_3L_1+\Delta L_3 L_1L_2+\Delta L_1 \Delta L_2L_3\nonumber\\
  &\quad\quad+\Delta L_2 \Delta L_3L_1+\Delta L_3 \Delta L_1L_2+\Delta L_3 \Delta L_1 \Delta L_2,\nonumber\\
  &\cong L_1L_2L_3+\Delta L_1 L_2L_3+\Delta L_2 L_3L_1+\Delta L_3 L_1L_2,
\end{align}
où nous avons négliger les termes $\Delta L_i^2$ et $\Delta L_i^3$ car $\Delta L_i\ll 1$.

On peut réexprimer cette équation 
\begin{equation}
V'-V=\Delta L_1 L_2L_3+\Delta L_2 L_3L_1+\Delta L_3 L_1L_2.
\end{equation}
Nous savons que la dilatation linéique nous donne $\Delta L_i=\alpha L_i\Delta T$ (ceci est vrai seulement si $\alpha$ est le même dans chaque direction $i$, et donc que le solide est isotrope). On a donc finalement
\begin{equation}
\Delta V'=\alpha L_1 L_2L_3\Delta T+\alpha L_1 L_2L_3\Delta T+\alpha L_1 L_2L_3\Delta T=3\alpha V\Delta T.
\end{equation}
En identifiant maintenant cette équation avec l'équation de la dilatation volumique vue en cours, il vient
\begin{equation}
\Delta V'=3\alpha V\Delta T=\beta V\Delta T.
\end{equation}
On voit donc que $\beta=3\alpha$.

\end{corrige}

\begin{exercice}{Fil de Cuivre}

\begin{enumerate}
	\item Un fil de cuivre ($\alpha=16.6\cdot 10^{-6}\K^{-1}$) est long de $10\ \m$ à la température de $20\oC$, quelle doit être sa
température pour qu'il s'allonge de $1\ \cm$ ?.
	\item La tour Eiffel, qui est en acier ($\alpha=11\cdot 10^{-6}\K^{-1}$), est de hauteur $320\ \m$ à $20\oC$, quelle est sa variation de hauteur
sur l’intervalle $-20$ à $+35\ \oC$ ?
\end{enumerate}
\end{exercice}

\begin{corrige}{Fil de Cuivre}
\begin{enumerate}
	\item 
\begin{equation}
\Delta T=\frac{\Delta L}{\alpha_\mathrm{cu}L}=\frac{10^{-2}}{16.6\cdot10^{-6}10}=60.2^\circ\C.
\end{equation}
Ainsi la température doit être de $20+60.2=80.2^\circ\C$
\item De $20\oC$ à $-20\oC$ on a 
\begin{equation}
\Delta L=\Delta T \alpha_\mathrm{acier}L=11\cdot10^{-6}\cdot 320\cdot(-40)=-0.141\m.
\end{equation}
De $20\oC$ à $35\oC$ on aura
\begin{equation}
\Delta L=\Delta T \alpha_\mathrm{acier}L=11\cdot10^{-6}\cdot 320\cdot15=0.0528\m.
\end{equation}
Finalement entre $-20\oC$ et $35\oC$ on aura $\Delta L=0.0528-(-0.141)=0.194\m$.
\end{enumerate}
\end{corrige}

\begin{exercice}{Thermomètre}

Un thermomètre à mercure ($\beta=1.8\cdot 10^{-4}\K^{-1}$) en quartz est formé par un volume total de $0.400\ \cm^3$ de mercure
dans un réservoir surmonté par un tube de quartz. Le tube a un diamètre de $0.20\ \mm$.
On négliger la dilatation volumique du quartz, de combien monte le niveau du mercure
lorsque la température passe de $10\oC$ à $90\oC$ ?
\end{exercice}

\begin{corrige}{Thermomètre}

Lorsque la température passe de $10\oC$ à $90\oC$, la variation de volume du mercure est donnée par 
\begin{equation}
\Delta V=\Delta T \beta_\mathrm{hg}V_\mathrm{hg}=(90-10)\cdot 1.8\cdot 10^{-4}\cdot 400=5.76\mm^3.
\end{equation}
Dans le tube de rayon $0.10\mm$ et donc de section $S=\pi 0.1^2=0.0314\mm^2$.
Le liquide montera donc de la hauteur  
\begin{equation}
h=\frac{\Delta V}{S}=\frac{5.76}{0.0314}=183\mm.
\end{equation}

\end{corrige}

\begin{exercice}{Débordement}

Un bêcher en Pyrex ($\beta_\mathrm{Pyrex}=9\cdot10^{-6}\K^{-1}$) est rempli jusqu'au bord avec $100\ \cm^3$ d'eau ($\beta_\mathrm{eau}=2\cdot10^{-4}\K^{-1}$) à $10\oC$.
Quel volume d'eau déborde si la température atteint $50\oC$ ?

Remarque : il faut tenir compte de la dilatation du verre.
\end{exercice}

\begin{corrige}{Débordement}

Lorsque la température passe de $10\oC$ à $50\oC$, la variation de volume 
des $100\cm^3$ d'eau est donnée par 
\begin{equation}
\Delta V_\mathrm{eau}=\beta_{eau}V_\mathrm{eau}\Delta T=2\cdot 10^{-4}\cdot 100\cdot(50-10)=0.8\cm^3.
\end{equation}
Le bêcher en Pyrex augmente également de volume
\begin{equation}
\Delta V_\mathrm{pyrex}=\beta_{pyrex}V_\mathrm{pyrex}\Delta T=9\cdot 10^{-6}\cdot 100\cdot(50-10)=0.036\cm^3.
\end{equation}
Ainsi l’eau déborde de $\Delta V_\mathrm{eau}-\Delta V_\mathrm{pyrex}=0.764\cm^3$.
\end{corrige}

\begin{exercice}{Pression des pneus}

On gonfle un pneu d’automobile dont le volume intérieur
(de 15 litres) reste pratiquement constant à $15\oC$ à la pression de $300\ \mathrm{kPa}$.
Après un voyage on constate que la pression est de $320\ \mathrm{kPa}$, quelle est alors la température de l’air à
l’intérieur du pneu ?
\end{exercice}

\begin{corrige}{Pression des pneus}
    
La loi des gaz parfaits nous dit
\begin{equation}
\frac{p}{T}=\frac{nR}{V}.
\end{equation}
Ici le volume $V$ (de 15 litres) reste constant, et la quantité de gaz à l’intérieur $n$ est également constante.

Ainsi 
\begin{equation}
\frac{p}{T}=cte,
\end{equation}
et donc
\begin{equation}
\frac{p_1}{T_1}=\frac{p_2}{T_2}.
\end{equation}
Finalement,
\begin{equation}
T_2=\frac{p_2T_1}{p_1}=\frac{(273+15)\cdot 320\cdot 10^3}{300\cdot 10^3}=307.2^\circ\K=34.2\oC.
\end{equation}

\end{corrige}


\end{document}