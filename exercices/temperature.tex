% Copyright (C)  2016  Orestis Malaspinas.
%     Permission is granted to copy, distribute and/or modify this document
%     under the terms of the GNU Free Documentation License, Version 1.3
%     or any later version published by the Free Software Foundation;
%     with no Invariant Sections, no Front-Cover Texts, and no Back-Cover Texts.
%     A copy of the license can be downloaded from:
%     https://www.gnu.org/licenses/fdl.html.

\documentclass[a4paper,12pt]{article}
\usepackage[utf8]{inputenc}
\usepackage[french]{babel}
\usepackage{amsfonts,bm,amsmath,amssymb,graphicx,amsthm}
\usepackage{cancel}
\usepackage{mathtools}
\usepackage{caption}
\usepackage{hyperref}

\setlength{\parindent}{0pt}

\newcommand{\real}{\mathbb{R}}
\newcommand{\integer}{\mathbb{Z}}
\renewcommand{\natural}{\mathbb{N}}
\newcommand{\complex}{\mathbb{C}}
\newcommand{\zbar}{\bar{z}}
\newcommand{\dd}{\mathrm{d}}
\newcommand{\perm}{\mathrm{perm}}
\newcommand{\card}{\mathrm{card}}
\newcommand{\fh}{\hat{f}}
\newcommand{\gh}{\hat{g}}
\newcommand{\hh}{\hat{h}}
\renewcommand{\Re}{\mathrm{Re}}
\renewcommand{\Im}{\mathrm{Im}}
\newcommand{\pDeriv}[2]{\frac{\partial #1}{\partial #2}}
\newcommand{\pDerivTwo}[2]{\frac{\partial^2 #1}{\partial #2^2}}
\newcommand{\dDeriv}[2]{\frac{\dd #1}{\dd #2}}
\newcommand{\dDerivTwo}[2]{\frac{\dd^2 #1}{\dd #2^2}}
\newtheorem{definition}{Définition}
\newtheorem*{exemples}{Exemples}
\newtheorem*{exemple}{Exemple}
\newtheorem*{solution}{Solution}
\newtheorem{exercice}{Exercice}
\newtheorem{corrige}{Corrig\'e exercice}
\newtheorem*{remarque}{Remarque}
\newtheorem{proprietes}{Propriétés}
\newtheorem{theoreme}{Théorème}
\newcommand{\cm}{\mathrm{cm}}
\newcommand{\km}{\mathrm{km}}
\newcommand{\mm}{\mathrm{mm}}
\newcommand{\cd}{\mathrm{cd}}
\newcommand{\mol}{\mathrm{mol}}
\newcommand{\m}{\mathrm{m}}
\newcommand{\s}{\mathrm{s}}
\newcommand{\kg}{\mathrm{kg}}
\newcommand{\g}{\mathrm{g}}
\newcommand{\K}{\mathrm{K}}
\newcommand{\J}{\mathrm{J}}
\newcommand{\C}{\mathrm{C}}
\newcommand{\oC}{^\circ\C}
\newcommand{\A}{\mathrm{A}}
\newcommand{\N}{\mathrm{N}}
\newcommand{\atm}{\mathrm{atm}}
\renewcommand{\bar}{\mathrm{bar}}
\newcommand{\V}{\mathrm{V}}
\newcommand{\W}{\mathrm{W}}
\newcommand{\Pa}{\mathrm{Pa}}


\title{Exercices sur la température}
% \author{Orestis Malaspinas}
\date{}

\begin{document}
\maketitle

\begin{exercice}{Dilatation linéique/volumique}

Montrer qu’en première approximation, pour un solide sa dilatation volumique est telle que 
\begin{equation}
\beta=3\alpha.
\end{equation}
Indication: Écrire que l'allongement dans chaque direction peut s'écrire $L_1=L_0+\Delta L$, avec $\Delta L\ll 1$,
en déduire la variation
du volume, puis négliger les termes les plus petits.
\end{exercice}

\begin{exercice}{Fil de Cuivre}

\begin{enumerate}
	\item Un fil de cuivre ($\alpha=16.6\cdot 10^{-6}\K^{-1}$) est long de $10\ \m$ à la température de $20\oC$, quelle doit être sa
température pour qu'il s'allonge de $1\ \cm$ ?.
	\item La tour Eiffel, qui est en acier ($\alpha=11\cdot 10^{-6}\K^{-1}$), est de hauteur $320\ \m$ à $20\oC$, quelle est sa variation de hauteur
sur l’intervalle $-20$ à $+35\ \oC$ ?
\end{enumerate}
\end{exercice}

\begin{exercice}{Thermomètre}

Un thermomètre à mercure ($\beta=1.8\cdot 10^{-4}$) en quartz est formé par un volume total de $0.400\ \cm^3$ de mercure
dans un réservoir surmonté par un tube de quartz. Le tube a un diamètre de $0.20\ \mm$.
On négliger la dilatation volumique du quartz, de combien monte le niveau du mercure
lorsque la température passe de $10\oC$ à $90\oC$ ?
\end{exercice}

\begin{exercice}{Débordement}

Un bêcher en Pyrex ($\beta_\mathrm{Pyrex}=9\cdot10^{-6}\K^{-1}$) est rempli jusqu'au bord avec $100\ \cm^3$ d'eau ($\beta_\mathrm{eau}=2\cdot10^{-4}\K^{-1}$) à $10\oC$.
Quel volume d'eau déborde si la température atteint $50\oC$ ?

Remarque : il faut tenir compte de la dilatation du verre.
\end{exercice}

\begin{exercice}{Pression des pneus}

On gonfle à $15\oC$ à la pression de $300\ \mathrm{kPa}$ un pneu d’automobile dont le volume intérieur
(de 15 litres) reste pratiquement constant.
Après un voyage on constate que la pression est de $320\ \mathrm{kPa}$, quelle est alors la température de l’air à
l’intérieur du pneu ?
\end{exercice}

% \section{Corrigé}

% \begin{corrige}

% Soit $V=$ et après dilatation, on a
% $\Delta L_$
% : et en dilatation , et ainsi~:

% Si alors on \emph{\textbf{peut négliger le terme 2}} toujours beaucoup\\
% plus petit que le terme 1~!

% Ainsi

% Mais en dilatation , et , et ainsi~:

% Soit~: et\\
% \end{corrige}

% Exercice 3

% \begin{enumerate}
% \def\labelenumi{\alph{enumi})}
% \item
%   ,\\
%   ainsi la température doit être de
% \item
%   De à -20°C~ on aura~:
% \end{enumerate}

% \begin{quote}
% De à +35°C~ on aura~:

% Et de à +35°C~ on aura~:
% \end{quote}

% Exercice 4

% Lorsque la température passe de à a variation de volume du mercure est
% donnée par~:

% Dans le tube de rayon et donc de section ,

% Le liquide montera de la hauteur

% Exercice 5

% Lorsque la température passe de à , la variation de volume\\
% des 100 cm\textsuperscript{3} d'eau est donnée par~:

% Le bêcher en Pyrex augmente également de volume, son coefficient de
% dilatation volumique étant donné par~: on aura~:

% Ainsi l'eau déborde de

% Exercice 6

% De la loi des gaz parfaits on a~:

% Ici le volume (de ) reste constant, et la quantité de gaz à l'intérieur
% est également constante.

% Ainsi~: , d'où °C~


\end{document}